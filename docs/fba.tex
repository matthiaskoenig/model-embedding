\chapter{Flussbilanzanalyse [FBA]}
\label{fba}
\section{Einleitung}
Modelle basierend auf gewöhnlichen Differentialgleichungen (ODE) ermöglichen es, dynamische Zustandsänderungen von Zellen zu untersuchen. Die Modellierung großer metabolischer Netzwerke mittels ODE gestaltet sich allerdings als schwierig, da die mechanistischen Details und die kinetischen Parameter der Einzelprozesse nur in seltenen Fällen bekannt sind \cite{Stelling2002}. Weiterhin sind die für die Modellierung benötigten experimentellen Metabolitprofile und Flussdaten noch nicht ausreichend genug vorhanden, um prediktive kinetische Modelle großer Netzwerke aufzustellen \cite{Lee2006}.

Daher werden alternative Methoden zur Analyse großer metabolischer Netzwerke benötigt. Die FBA ist eine solche Methode, die weder kinetischen Parameter noch Metabolitkonzentrationen benötigt, sondern zunächst mit der stöchiometrischen Matrix des Systems auskommt. Sie ist eine weit verbreitete Methode zur Berechnung stationärer Flüsse in großen Netzwerken und wurde bereits zur Untersuchung vieler metabolischer Netzwerke von der Größe des Genoms (genome-scale) verwendet \cite{Duarte2004, Feist2007, Boelling2009}. Prediktive Aussagen mittels FBA sind für Prokaryoten wie \textit{S. aureus} \cite{Heinemann2005} oder \textit{E. coli} \cite{Mahadevan2002} über Eukaryoten wie \textit{S. cerevisiae} \cite{Duarte2004} bis hin zu humanen Zelltypen wie Erythrozyt \cite {Holzhuetter2004} oder Hepatozyt \cite{Boelling2009} möglich.

Die FBA ist eine mathematische Methode um Flussverteilungen im Fliessgleichgewicht in Netzwerken zu berechnen, die zusätzliche an das System gestellte Bedingungen (constraint) erfüllen müssen. Die Flusslösungen werden dabei bezüglich einer vorgegebenen Zielfunktion optimiert.\\
Den möglichen Phänotypen eines biologischen Systems sind Bedingungen auferlegt \cite{Covert2003, Price2004}. Diese Bedingungen sind physikalische Gesetze wie Massen- und Energieerhaltung, topologische Bedingungen wie die Beschränkung von Metaboliten auf bestimmte Kompartimente oder umweltbedingte Bedingungen wie die Verfügbarkeit von Ressourcen \cite{Lee2006}. Durch diese Einschränkungen wird der Raum der möglichen Flusslösungen begrenzt.\\
Dennoch existieren im Allgemeinen immer noch unendlich viele unterschiedliche Flusslösungen. Um eine eindeutige Lösung zu erhalten werden die mit den Nebenbedingungen verträglichen Flusslösungen bezüglich einer Zielfunktion optimiert. Beispiele für Zielfunktionen sind die Maximierung des Zellwachstum oder der Biomasseproduktion oder Minimierung des Gesamtflusses des Systems.

Die FBA ist auf der einen Seite ein effektives Werkzeug, um große biologische Netzwerke zu untersuchen und experimentell testbare Vorhersagen zu treffen \cite{Lee2006}. So kann beispielsweise leicht die Auswirkung von Störungen auf die Flusslösungen untersucht werden. Die Folgen einer Gendeletion können simuliert werden, indem der Fluss durch die Reaktionen, die von dem Gen kodiert werden, verboten wird. Inhibitorstudien können auf ähnliche Weise durchgeführt werden, indem der Fluss durch die inhibierten Reaktionen auf einen Bruchteil des normalen Flusses beschränkt wird. Die veränderten Flusslösungen erlauben Vorhersagen über die Folgen der Systemveränderung.\\
Auf der anderen Seite sind die Ergebnisse in weitem Maße hypothetischer Natur, da die Methode auf plausiblen aber schwer überprüfbaren Optimalitätsprinzipien beruhen \cite{Hoppe2007}.

Gute Übersichten über die Methode der FBA sind Oberhardt et al. \cite{Oberhardt2009}, Lee et al. \cite{Lee2006} oder Kauffman et al \cite{Kauffman2003}.

%%%	METHODE		%%%
\section{Theorie und Methode}
Um Flussverteilungen mittels FBA zu berechnen, werden die stöchiometrische Matrix $S$ des zu untersuchenden Systems, zusätzliche Bedingungen und eine definierte Zielfunktion $Z$ benötigt. Die FBA geht davon aus, dass das metabolische Netzwerk einen Zustand im Fliessgleichgewicht erreicht, der durch die vorgegebenen Bedingungen beschränkt wird. Vorhersagen für die Flusswerte erhält man durch Optimierung der Zielfunktion $Z$ bei gleichzeitiger Erfüllung der Bedingungen. Die Methodik der FBA lässt sich in vier Schritte zerlegen \cite{Kauffman2003}: 
\begin{itemize}
 \item Modelldefinition
 \item Stöchiometrische Matrix und Bedingung des Fliessgleichgewichts
 \item Definition zusätzlicher Bedingungen
 \item Zielfunktion
\end{itemize}

\begin{figure}[h]
 \centering
 \includegraphics[width=\textwidth,keepaspectratio=true]{./2_reconstruction/figures/fba_overview_small.png}
 \caption{Methode der FBA (aus Kauffman et al. \cite{Kauffman2003})
  \textbf{(a)} Ein Modellsystem bestehend aus drei Metaboliten (A, B und C) mit drei Reaktionen (interne Flüsse, $v_i$, mit einer reversiblen Reaktion) und drei Austauschflüssen ($b_i$). \textbf{(b)} Zeitliche Änderung der Metabolitkonzentrationen in Abhängigkeit von den Flüssen durch Reaktionen und Transporter. Die Gleichungen können in Matrixform geschrieben werden. Im Fliessgleichgewicht vereinfacht sich dies zu $S v = 0$. \textbf{(c)} Die Flüsse des Systems werden mittels Bedingungen beschränkt. Dies erzeugt einen Flusskegel \cite{Schilling2000, Papin2002}, der der metabolischen Kapazität des Systems entspricht.\textbf{(d)} Optimierung des Systems mit verschiedenen Zielfunktionen $Z$. Fall I liefert einen eindeutigen optimale Lösung, dagegen liefert Fall II mehrere optimale Lösungen entlang einer Kante.}
 \label{fig: fba_method}
\end{figure}

\subsection{Modelldefinition}
Als Ausgangspunkt der FBA muss das metabolische Netzwerk aufgestellt werden. Dies geschieht mittels metabolischer Netzwerkrekonstruktion (s. \ref{reconstruction}) und Erzeugung eines zu untersuchende Modellsystems bestehend aus Metaboliten und Prozessen (Reaktionen und Transporter) basierend auf der Rekonstruktion (Abb.~\ref{fig: fba_method} \textbf{a}).

\subsection{Stöchiometrische Matrix und Fliessgleichgewicht}
Die zeitlichen Veränderungen der Metabolitkonzentrationen wird mittels der stöchiometrischen Matrix $S$ und der Flüsse durch die Reaktionen und Transporter $v$ beschrieben (Abb.~\ref{fig: fba_method} \textbf{b}). Dies führt auf ein System gekoppelter ODEs der Form:
\begin{equation}
 % d has to be set different than the C
 \frac{dC}{dt} = Sv
\end{equation}
$C$ ist der Vektor der Metabolitkonzentrationen, $v$ der Vektor der Reaktionsflüsse, $t$ die Zeit. $S$ ist die stöchiometrische Matrix, wobei die Zeilen den Metaboliten und die Spalten den Reaktionen des metabolischen Netzwerks entsprechen. 

Die Annahme des Fliessgleichgewicht bildet die erste Menge grundlegender Bedingungen, die die Flusslösungen einschränkt. Sämtliche internen Metabolite sind im Fliessgleichgewicht bilanziert und dürfen sich weder anhäufen, noch verarmen. Die zeitlichen Änderungen der Konzentrationen verschwinden:
\begin{equation}
 \frac{dC}{dt} = 0
\end{equation}
Als erlaubte Flussverteilungen ergibt sich die Menge der Flussvektoren $v$, die die Gleichung 
\begin{equation}
 Sv = 0
\end{equation}
erfüllt.

Bei exakter Betrachtung sind Zellen keine Systeme, die sich im Fliessgleichgewicht befinden.
Die Annahme des Fliessgleichgewichts ist allerdings näherungsweise gültig, da die meisten intrazellulären Reaktionen typischerweise viel schneller ablaufen als sich der zellulären Phänotyp verändert \cite{Lee2006}.

\subsection{Zusätzliche Bedingungen}
 Zusätzliche Bedingungen sind notwendig, um die Menge der Lösungen weiter einzuschränken. (Abb.~\ref{fig: fba_method} \textbf{c}). Diese können beispielsweise thermodynamischer oder experimenteller Natur sein. Bedingungen auf Basis experimentell bestimmter Flussdaten führen zu Gleichungen der Form
\begin{equation}
 v_i^{l} \leq v_i \leq v_i^{u}
\end{equation}
wobei $v_i^{l}$ die experimentell bestimmte untere Grenze, $v_i^{u}$ die experimentell bestimmte obere Grenze des Flusses $v_i$ ist.

Zwangsbedingungen auf Grund von Reaktionsirreversibilitäten führen auf Gleichungen der Form
\begin{equation}
0 \leq v_i < \infty 
\label{eqn:irreversibel}
\end{equation}

\subsection{Zielfunktion}
Die Bedingungen des Fliessgleichgewichts und die zusätzlichen Bedingungen bilden ein unterbestimmtes lineares Gleichungssystem, da im allgemeinen mehr Reaktionen als Metabolite in den Netzwerken existieren. Durch die Bedingungen wird ein Flusskegel möglicher Lösungsflüsse erzeugt, der der metabolischen Kapazität des Systems entspricht \cite{Schilling2000, Papin2002}. Um eine eindeutige Lösung aus diesem Flusskegel zu selektieren, wird eine Optimierung bezüglich einer gegebenen Zielfunktion $Z$ durchgeführt.

Das unterbestimmte System wird hierzu als Optimierungsproblem formuliert (Abb.~\ref{fig: fba_method} \textbf{d}). Im Falle linearer Zielfunktionen führt dies auf ein Problem der linearen Programmierung (LP). Die optimale Lösung kann für LP Probleme mit heutigen Standardverfahren in polynomialer Zeit bestimmt werden. Für nichtlineare Zielfunktionen existieren entsprechende Verfahren, beispielsweise die quadratische Programmierung (QP) für quadratische Zielfunktionen. Für praktische Anwendungen existieren Algorithmen die LP und QP Probleme sehr schnnell lösen, so dass die FBA ohne Probleme auch auf sehr große Netzwerke angewandt werden kann.

Die Ergebnisse der FBA sind sehr stark von der gewählten Zielfunktion abhängig und für die Untersuchung metabolischer Netzwerke wurde eine Vielzahl unterschiedlicher Zielfunktionen verwendet. 
Zielfunktionen sind beispielsweise die Maximierung der Biomasse oder des Zellwachstums \cite{Mahadevan2002, Dien2002}, Maximierung von Reduktionskraft oder ATP \cite{Ramakrishna2001}, Maximierung der Syntheserate eines bestimmten Produktes \cite{Varma1993} oder Flussminimierung \cite{Holzhuetter2004}. Für mikrobielle Systeme konnte gezeigt werden, dass bei Verwendung einer Wachstumsfunktion als Zielfunktion viele FBA Vorhersagen konsistent mit den experimentellen Daten sind \cite{Ibarra2002, Burgard2003,Edwards2000}. Für einige Bedingungen stimmen die Vorhersagen allerdings nicht überein \cite{Ibarra2002, Segre2002, Burgard2003}.

\paragraph{Flussminimierung als Zielfunktion}
Für diese Arbeit wurde die in unserer Arbeitsgruppe entwickelte Methode der Flussminimierung zur Optimierung verwendet \cite{Holzhuetter2004}. In der Flussminimierung werden die stationären metabolischen Flüsse für vorgegebene Zielflüsse und gegebene externe Metabolite, die mit dem System ausgetauscht werden können, minimiert. 

Der Hepatozyt hat viele unterschiedliche Aufgaben, wie beispielsweise Homöostase des Blutglukosespiegels, Ammoniakentgiftung oder Ketonkörpersynthese. Zwischen diesen Aufgaben muss das System in Abhängigkeit von externen Ressourcen und Signalen umschalten.\\
Die verschiedenen Phänotypen beruhen letztendlich auf den metabolischen Flüssen durch die chemischen Reaktionen und Transportprozesse. Nahezu alle Reaktionen werden durch Enzyme in der Zelle katalysiert, die meisten der Transportprozesse durch Transportproteine ermöglicht. Die Aktivität dieser Proteine und damit der Fluss durch die zugehörigen metabolischen Prozesse kann durch Regulation auf unterschiedliche zelluläre Bedürfnisse angepasst werden, beispielsweise durch allosterische Effektoren, Proteinmodifikationen wie Phosphorylierungen oder Veränderung der Genexpression.\\
Da die Herstellung einer enzymatischen Kapazität für eine bestimmte zelluläre Funktionalität ein großer Aufwand für die Zelle ist wurde dieser Aufwand vermutlich evolutionär minimiert. D.h. die verschiedenen Aufgaben der Zelle werden jeweils mittels minimaler Flüsse realisiert.

Der zu minimierende Gesamtfluss wird als gewichtete Linearkombination aller Einzelflüsse gemessen, wobei die thermodynamischen Gleichgewichtskonstanten der Reaktionen $K_{eq}$ als Gewichtungsfaktoren verwendet werden. Also je weiter das Gleichgewicht in Richtung der Produkte verschoben ist, desto größer wird der Gewichtungsfaktor für die Rückwärtsreaktion. Flüsse entgegen der durch die Gleichgewichtskonstante vorgegebene Richtung werden bestraft, da die Erhaltung der notwendigen Konzentrationsgradienten, um einen solchen Fluss zu ermöglichen einen zusätzlichen zellulären Aufwand bedeutet.

Die Flussraten, die durch die Flussminimierungsmethode vorhergesagt werden, zeigen gute Korrelationen mit den Flussraten, die durch kinetische Modellierung oder durch direkte experimentelle Messungen erhalten werden \cite{Holzhuetter2004}.\\
Größere Abweichungen treten allerdings für Bereiche des Netzwerks auf, die redundante Wege darstellen, da hier die FBA immer die bezüglich der Flussminimierung günstigste Route wählt \cite{Holzhuetter2004}. Diese Lösung muss so nicht in der Zelle realisiert sein. Stattdessen können in der Zelle Kombinationen unterschiedlicher Routen existieren oder durch Regulationen Reaktionspfade gewählt werden, die bezüglich der Flussminimierung nicht optimal sind.

Für eine detaillierte Herleitung und Beschreibung der Methodik siehe Holzhütter \cite{Holzhuetter2004}.

\paragraph{Thermodynamische Zusatzbedingungen}
Thermodynamische Eigenschaften als zusätzliche Bedingungen für die FBA wurden in den letzten Jahren in mehreren Arbeiten verwendet \cite{Feist2007, Henry2007, Kuemmel2006} und auch in unserer Arbeitsgruppe wurde eine Methode zur thermodynamischen Realisierbarkeit entwickelt \cite{Hoppe2007}. 
Für die in dieser Arbeit durchgeführten FBA Simulationen mittels Flussminimierung wurden die thermodynamischen Daten von Feist et al. \cite{Feist2007} verwendet. Die freien Gibbschen Reaktionsenthalpien $\bigtriangleup G_r^{°'}$ wurden mittels Gruppenübertragungs-Theorie \cite{Jankowski2008} basierend auf der Methode von Mavrovouniotis \cite{Mavrovouniotis1990, Mavrovouniotis1991} berechnet.
Die Umrechnung in die Gleichgewichtskonstanten erfolgte nach
\begin{equation}
\bigtriangleup G_r^{°'} = - RT \ln(K_{eq})
\end{equation}
mit $T=310.15 \unit{K}$ und $R=8.314 \unit{\frac{J}{mol K}}$.

\subsection{Software}
FBA ist nicht besonders rechenintensiv und kann auf Standard Desktop PCs durchgeführt werden. Für die Simulationen wurde die in dieser Arbeitsgruppe entwickelte FBA Simulationssoftware Fasimu verwendet, die auf dem linearen Solver ILOG CPLEX \cite{cplex} aufbaut.


\subsection{Simulationen}
Sämtliche Simulationen wurden mit dem Modell des Kernhepatozyten basierend auf der abgeschlossenen Netzwerkrekonstruktion durchgeführt.

Einige wenige Metabolite mussten in den Simulationen die Fliessgleichgewichtsbedingungen nicht erfüllen. Dies war Wasser in allen Kompartimenten, sowie Phosphat im Zytosol. Wasser und Phosphat liegen in hohen Konzentration vor, Phosphat steht weiterhin über Puffersysteme im Zytosol für Reaktionen zur Verfügung.

In allen FBA Simulationen wurden neben den Bedingungen zum Fliessgleichgewicht zusätzliche Irreversibilitätsbedingungen eingeführt (s. Gleichung \ref{eqn:irreversibel}), hauptsächlich drei Kategorien zuzuordnen sind: Decarboxylierungen, Pyrophosphatreaktionen und Reaktionen mit energieabhängiger Umkehrreaktion.

\paragraph{Decarboxylierungen}
Decarboxylierungen sind in vielen Fällen irreversibel. Die Umkehrreaktion der Decarboxylierung, die Carboxylierung, benötigt meist spezielle Kofaktoren wie Biotin oder Vitamin K, um $\text{CO}_2$ in organische Moleküle einzubauen, wie beispielsweise die Pyruvat Carboxylase (PC), eine wichtigen biotinabhängige Reaktion der Gluconeogenese und anaplerotische Reaktion des Citratzyklus \cite{Jitrapakdee2008}.\\
Zwei Beispiele dieser irreversiblen Decarboxylierungen sind der Pyruvat Dehydrogenase Komplex (PDHc) und der $\alpha$-Ketoglutarat Dehydrogenase Komplex (KGDHc) \cite{Reisch2007}.\\
Die Irreversibilität der Gesamtreaktion des PDHc konnte durch Isotopenmarkierungsexperimente gezeigt werden:. Der Komplex ist nicht in der Lage radioaktiv markiertes $\text{CO}_2$ an Acetyl-CoA zu binden \cite{Nelson2008}. Der KGDHc katalysiert die irreversible Decarboxylierung von $\alpha$-Ketoglutarat zu Succinyl-CoA. Ähnlich zur PHDc ist auch die KGDHc eine Multienzymkomplex bestehend aus mehreren Kopien von drei Proteinuntereinheiten und es wurde gezeigt, dass die Reaktion nur in Richtung der Decarboxylierung abläuft \cite{Reisch2007}. 

Für sämtliche Reaktionen in denen $\text{CO}_2$ oder $\text{HCO}_3^-$ als Metabolit beteiligt ist, wurde bezüglich der Irreversibilität recherchiert und die gefundenen zusätzliche Irreversibilitätsbedingungen für die Simulationen verwendet.

\paragraph{Pyrophosphat abhängige Reaktionen}
Pyrophosphat ($\text{PP}_i$) entsteht als Nebenprodukt in einigen enzymatischen Reaktionen. Dies sind beispielsweise Reaktionen der DNA und RNA Synthese, die Aktivierung von Fettsäuren, Aminosäuren und Zuckern oder die Synthese der zyklischen Nukleotide cAMP und cGMP.\\
$\text{PP}_i$ kann nicht effizient über Membranen transportiert werden und die schnelle intrazelluläre Hydrolyse in anorganisches Phosphat durch Pyrophosphatasen (PPase) ist wichtig für den thermodynamischen Zug dieser Reaktionen \cite{Curbo2006}. Durch die geringe $\text{PP}_i$ Konzentration wird das Gleichgewicht stark auf die Seite der $\text{PP}_i$ Bildung verschoben.\\
Alle $\text{PP}_i$ abhängigen Reaktionen wurden in den Simulationen daher als irreversibel in Richtung der $\text{PP}_i$ Produktion betrachtet (mit Ausnahme der PPase).

\paragraph{Reaktionen mit energieabhängiger Umkehrreaktion}
Die dritte Klasse irreversibler Reaktionen sind energieabhängige Reaktionen, für die eine energieunabhängige Umkehrreaktion zwischen Hauptsubstraten und -produkten existiert. In den Simulationen wurden die energieabhängige Reaktion samt Umkehrreaktion als irreversibel behandelt.\\
Ein solches Paar ist beispielsweise die Glucokinase (GK), die die ATP abhängige Phosphorylierung der Glucose katalysiert
\begin{equation*}
\text{ATP} + \text{Glucose} \rightarrow \text{Glucose-6P} + \text{ADP}
\end{equation*}
und die energieunabhängige Umkehrreaktion der Glucose-6 Phosphatase (G6P)
\begin{equation*}
\text{Glucose-6P} + \text{H}_2\text{O} \rightarrow \text{Glucose} + \text{P}_i
\end{equation*}.

Allein basierend auf der Thermodynamik der Reaktionen ist ein geringer Fluss in Umkehrrichtung möglich. In dieser Arbeit wird davon ausgegangen, dass diese Umkehrflüsse keinen wesentlichen Beitrag zu dem von der Zelle benötigten Fluss von Produkten zu Substraten leisten.\\
Die so definierten Irreversibilitäten sind nur Näherungen und in der Zelle kann die Umkehrreaktion in sehr geringem Maße stattfinden. Falls die Reaktionsrichtungen nicht eingeschränkt werden, werden in den FBA Simulationen durch solche Reaktionspaare sehr kurze energieproduzierende Zyklen möglich. Die Kombination der Glucokinase und der Glucose-6 Phosphatase in jeweiliger Umkehrrichtung ermöglicht es in einem Zweierzyklus ATP zu erzeugen ($\text{ADP} + \text{P}_i \rightarrow \text{ATP}$). Diese kurzen Zyklen werden in der Flussminimierung bevorzugt ausgewählt, da Bereitstellung von ATP durch andere Stoffwechselwege im Vergleich dazu bezüglich des Flussminimierungskriteriums deutlich aufwendiger ist.\\
Da die energieunabhängigen Umkehrreaktionen zumeist eine hohen negative Standardreaktionsenthalpie haben (beispielsweise Hydrolyse von Glucose-6P durch G6P $\bigtriangleup G_{r}^{'0} = 13.8 \frac{\unit{kJ}}{\unit{mol}}$ \cite{Nelson2008}) sind diese Umkehrreaktionen in der Flussminimierung recht teuer. Die Bildung von ATP benötigt allerdings normalerweise viele Reaktionen (Glykolyse bzw. Citratzyklus mit oxidative Phosphorylierung), so dass diese kurzen ATP Zyklen dennoch in der Optimierung bevorzugt werden, auch wenn sie keinerlei Bedeutung für die Energieversorgung der Zelle besitzen.

Daher wurden solche Reaktionspaare in den Simulationen als irreversibel betrachtet. Weitere Beispiele neben GK und G6Pase sind Phosphofructokinase 1 und 2 (PFK1, PFK2) mit den zugehörigen Umkehrreaktionen Fructose-6 Phosphatase 1 und 2 (FBP1, FBP2).

\paragraph{Weitere irreversible Reaktionen}
Weiterhin führten Reaktionen, für die eindeutige experimentelle Daten zur Irreversibilität vorlagen, zu weiteren Irreversibilitätsbedingungen in den Simulationen.


\subsection{Einzelsimulation}
Eine einzelne FBA  Simulation wird durch die in der Simulation austauschbaren Metabolite, die Zielflüsse der Simulation, sowie eventuell zusätzliche für die Simulation geltende Nebenbedingungen definiert.

\paragraph{Zielflüsse}
Zielflüsse sind Flüsse, die Bestandteil der Flusslösung sein müssen. In nahezu allen Simulationen wird nur ein einzelner Zielfluss vorgegeben. Der Zielfluss in der Simulation zur Glykogenproduktion ist beispielsweise die Glykogensynthase. Sämtliche Flusswerte in den Simulationen sind relative Flusswerte bezogen auf den angegebenen Zielfluss, der auf den Wert 1 normiert wurde.

\paragraph{Austauschbare Metabolite}
Für jede Simulation müssen die vom Hepatozytenmodell mit der Umgebung (Blut) austauschbaren Metabolite festgelegt werden. In der Simulation zum Abbau von Glucose zu Laktat wird beispielsweise nur der Import von Glucose und der Export von Laktat erlaubt.\\
In einem sog. minimalen Austauschset wird für das Hepatozytenmodell die Austauschkapazität mit dem Blut definiert. Dieses wird für einige Simulationen benötigt, wie beispielsweise in der Simulation zur Synthese von Proalbumin. Darin enthalten sind die Metabolite, die aus der Blutbahn aufgenommen werden können (Glucose, die essentiellen Aminosäuren Phenylalanin, Valin, Threonin, Trypthophan, Isoleucin, Methionin, Leucin und Lysin, sowie Folat und Sauerstoff) und die Metabolite, die vom Hepatozyten an das Blut abgegeben werden können (Urea, Urate, Glucose und die Ketonkörper Acetoacetat, Aceton und $\beta$-Hydroxybutyrat, sowie $\text{CO}_2$). Auch die Produktion von Glykogen ist in dem minimalen Austauschset erlaubt, dieses wird allerdings nicht an das Blut abgegeben, sondern im Hepatozyten gespeichert.

\paragraph{Zusätzliche Simulationsbedingungen}
Für die einzelnen Simulation gelten zum Teil Zusatzbedingungen, wie Einschränkung von Reaktionsrichtungen oder Verbot von Flüssen durch bestimmte Reaktionen.\\
Die Fettsäuresynthese und die $\beta$-Oxidation werden gegenläufig reguliert (v.a. über Malonyl-CoA). Daher wurde bei den Simulationen zur Fettsäuresynthese der Fluss durch die $\beta$-Oxidation verboten und umgekehrt bei den Simulationen zur $\beta$-Oxidation die Fettsäuresynthase.\\
Ein weiteres Beispiel ist die Ketonkörpersynthese, die nur stattfindet, wenn die Acetyl-CoA Konzentration hoch ist, aber im Mitochondrium kein Oxalacetat für die Citrat Synthase (CS) Reaktion zur Verfügung steht. Die CS Reaktion wurde daher für die Ketonkörpersimulationen ausgeschlossen. Interessant ist, dass die Simulationen zur Ketonkörpersynthese erst die erwarteten Ergebnisse lieferten (HMG-CoA Synthase zu Acetoacetat und dann gegebenenfalls Umwandlung zu $\beta$-Hydroxybutyrat oder Aceton und Export der Ketonkörper), wenn neben Acetyl-CoA als verfügbarer Metabolit auch die CS ausgeschaltet wurde. Ansonsten wurde in den Flusslösungen die Citrat Synthase verwendet, um das Acetyl-CoA im Citratzyklus zu oxidieren.

\paragraph{Flusseinheiten}
\label{flux_units}
In allen durchgeführten FBA Simulation mittels Flussminimierung wird jeweils ein Zielfluss von 1 µmol/min/kg durch eine bestimmte Reaktion vorgegeben. Alle in den Simulationen angegebenen Flüsse sind Relativflüsse zu diesem Zielfluss und daher einheitenlos. 